\subsection*{Algoritmo da Divisão}
Se $x, y \in \mathbb{Z}$, com $y \neq 0$, então existem (e são únicos) $q, r \in  \mathbb{Z}$ com $0 \leq r < |b|$, tais que, $x = qy + r$.
\subsubsection*{Demonstração}
\begin{description}
    \item[(i)] Existência \newline
    \begin{description}
        \item[(Caso $b > 0$)]
        Consideremos o conjunto dos números múltiplos de $b$ ordenados de acordo com a ordem natural da reta, isto é, o conjunto

        $[\dots, -3b, -2b, -b, 0, b, 2b, 3b, \dots]$

        com,

        $\dots, -3b < -2b < -b < 0 < b < 2b < 3b, \dots$.

        Note que disso decorre uma decomposição da reta em intervalos disjuntos da forma

        $[qb, (q + 1)b] = {x \in \mathbb{R} : qb \leq x < (q + 1)b}$.

        com $q \in \mathbb{Z}$.

        Assim, dado $a \in \mathbb{Z}$, este pertence a apenas um desses intervalos e portanto necessariamente é da forma $a = qb + r$, com $q \in \mathbb{Z}$ e $r \geq 0$. É claro que $r < (q + 1)b - qb = b$.

        \item[(Caso $b < 0$)]
        Aplicamos o teorema no caso demonstrado em (i) para determinar $q_1, r \in \mathbb{Z}$, com $0 \leq r < |b|$ para escrever: $a = q_1|b| + r$. Fazendo $q = q_1$, como $|b| = -b$, (pois $b < 0$), obtemos de $a = q_1|b| + r$ que $a = qb + r$, onde $q, r \in \mathbb{Z}$ e $0 \leq r < |b|$.
    \end{description}
    \item[(ii)] Unicidade

    Resta demonstrar que $q$ e $r$, os quais satisfazem $a = qb + r$ são únicos.

    De fato, suponha que $a = qb + r$ e $a = q_1 b + r_1$, com $0 \leq r < |b|$ e $0 \leq r_1 < |b|$.

    Assim, $qb +  r = q_1 + r_1 \Rightarrow r-r_1 = (q_1 - q)b$.

    Afirmamos que $r = r_1$.

    Com efeito, se $r \neq r_1$, então: $0 < |r_1 - r|$.

    Além disso, $|r_1 - r| < b$. Pois, podemos admitir, sem perda de generalidade que $r < r_1$, consequentemente $r_1 - r > 0$ e $|r_1 - r| = r_1 - r$.

    Assim, se $r_1 - r = |b|$, então $r_1 = |b| + r$ e portanto $r_1 > |b|$, que é absurdo.

    Também, se $r_1 - r > |b|$, então $r_1 > |b| + r > |b|$, gerando novamente o absurdo $r_1 > b$.

    Logo, pela lei da tricotomia, $|r_1 - r| = r_1 - r < |b|$.

    Portanto, do acima segue que $0 < |r_1 - r| < |b|$.

    Agora, de $r-r_1 = (q_1 - q)b$ obtemos, $|r_1 - r| = |q_1 - q|b$.

    O que segue que $0 <  |q_1 - q|b < |b|$.

    Logo, $0 < |q_1 - q| < 1$, o que é um absurdo, pois $|q_1 - q|$ é um número inteiro (pois $q$ e $q_1 \in \mathbb{Z}$ e em $\mathbb{Z}$ vale a lei do Fechamento da Adição). Portanto $r = r_1$.

    Note que essa igualdade combinada com $|r_1 - r| = |q_1 - q|b$ implica $q_1 = q$, já que $0 = (q_1 - q)b$ e $b \neq 0$ por hipótese.
\end{description}